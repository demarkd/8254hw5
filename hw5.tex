\documentclass[english,letter,doublesided]{article}
\usepackage{rotating}
\newcommand{\G}{\overline{C_{2k-1}}}
\usepackage[latin9]{inputenc}
\usepackage{amsmath,calligra,mathrsfs,amsfonts}
\usepackage{amssymb}
\usepackage{lmodern}
\usepackage{mathtools}
\usepackage{enumitem}
\usepackage{pgf}
\usepackage{tikz}
\usepackage{tikz-cd}
\usepackage{relsize}

\usetikzlibrary{arrows, matrix}
%\usepackage{natbib}
%\bibliographystyle{plainnat}
%\setcitestyle{authoryear,open={(},close={)}}
\let\avec=\vec
\renewcommand\vec{\mathbf}
\renewcommand{\d}[1]{\ensuremath{\operatorname{d}\!{#1}}}
\newcommand{\pydx}[2]{\frac{\partial #1}{\newcommand\partial #2}}
\newcommand{\dydx}[2]{\frac{\d #1}{\d #2}}
\newcommand{\ddx}[1]{\frac{\d{}}{\d{#1}}}
\newcommand{\hk}{\hat{K}}
\newcommand{\hl}{\hat{\lambda}}
\newcommand{\ol}{\overline{\lambda}}
\newcommand{\om}{\overline{\mu}}
\newcommand{\all}{\text{all }}
\newcommand{\valph}{\vec{\alpha}}
\newcommand{\vbet}{\vec{\beta}}
\newcommand{\vT}{\vec{T}}
\newcommand{\vN}{\vec{N}}
\newcommand{\vB}{\vec{B}}
\newcommand{\vX}{\vec{X}}
\newcommand{\vx}{\vec {x}}
\newcommand{\vn}{\vec{n}}
\newcommand{\vxs}{\vec {x}^*}
\newcommand{\vV}{\vec{V}}
\newcommand{\vTa}{\vec{T}_\alpha}
\newcommand{\vNa}{\vec{N}_\alpha}
\newcommand{\vBa}{\vec{B}_\alpha}
\newcommand{\vTb}{\vec{T}_\beta}
\newcommand{\vNb}{\vec{N}_\beta}
\newcommand{\vBb}{\vec{B}_\beta}
\newcommand{\bvT}{\bar{\vT}}
\newcommand{\ka}{\kappa_\alpha}
\newcommand{\ta}{\tau_\alpha}
\newcommand{\kb}{\kappa_\beta}
\newcommand{\tb}{\tau_\beta}
\newcommand{\hth}{\hat{\theta}}
\newcommand{\evat}[3]{\left. #1\right|_{#2}^{#3}}
\newcommand{\prompt}[1]{\begin{prompt*}
		#1
\end{prompt*}}
\newcommand{\vy}{\vec{y}}
\DeclareMathOperator{\sech}{sech}
\DeclareMathOperator{\Spec}{\mathbf{Spec}}
\DeclareMathOperator{\spec}{Spec}
\DeclareMathOperator{\spm}{Spm}
\DeclareMathOperator{\rad}{rad}
\newcommand{\mor}{\mathrm{Mor}}
\newcommand{\obj}{\mathrm{Obj}~}
\DeclarePairedDelimiter\abs{\lvert}{\rvert}%
\DeclarePairedDelimiter\norm{\lVert}{\rVert}%
\newcommand{\dis}[1]{\begin{align}
	#1
	\end{align}}
\renewcommand{\AA}{\mathbb{A}}
\newcommand{\Aa}{\mathscr{A}}
\newcommand{\LL}{\mathcal{L}}
\newcommand{\CC}{\mathbb{C}}
\newcommand{\DD}{\mathbb{D}}
\newcommand{\RR}{\mathbb{R}}
\newcommand{\NN}{\mathbb{N}}
\newcommand{\ZZ}{\mathbb{Z}}
\newcommand{\QQ}{\mathbb{Q}}
\newcommand{\Ss}{\mathcal{S}}
\newcommand{\OO}{\mathcal{O}}	
\newcommand{\BB}{\mathcal{B}}
\newcommand{\Pcal}{\mathcal{P}}
\newcommand{\FF}{\mathbb{F}}
\newcommand{\Ff}{\mathscr{F}}
\newcommand{\Gg}{\mathscr{G}}
\newcommand{\PP}{\mathbb{P}}
\newcommand{\Fcal}{\mathcal{F}}
\newcommand{\Gcal}{\mathcal{G}}
\newcommand{\fsc}{\mathscr{F}}
\newcommand{\afr}{\mathfrak{a}}
\newcommand{\bfr}{\mathfrak{b}}
\newcommand{\cfr}{\mathfrak{c}}
\newcommand{\dfr}{\mathfrak{d}}
\newcommand{\efr}{\mathfrak{e}}
\newcommand{\ffr}{\mathfrak{f}}
\newcommand{\gfr}{\mathfrak{g}}
\newcommand{\hfr}{\mathfrak{h}}
\newcommand{\ifr}{\mathfrak{i}}
\newcommand{\jfr}{\mathfrak{j}}
\newcommand{\kfr}{\mathfrak{k}}
\newcommand{\lfr}{\mathfrak{l}}
\newcommand{\mfr}{\mathfrak{m}}
\newcommand{\nfr}{\mathfrak{n}}
\newcommand{\ofr}{\mathfrak{o}}
\newcommand{\pfr}{\mathfrak{p}}
\newcommand{\qfr}{\mathfrak{q}}
\newcommand{\rfr}{\mathfrak{r}}
\newcommand{\sfr}{\mathfrak{s}}
\newcommand{\tfr}{\mathfrak{t}}
\newcommand{\ufr}{\mathfrak{u}}
\newcommand{\vfr}{\mathfrak{v}}
\newcommand{\wfr}{\mathfrak{w}}
\newcommand{\xfr}{\mathfrak{x}}
\newcommand{\yfr}{\mathfrak{y}}
\newcommand{\zfr}{\mathfrak{z}}
\newcommand{\Dcal}{\mathcal{D}}
\newcommand{\Ccal}{\mathcal{C}}
\newcommand{\Ical}{\mathcal{I}}
\usepackage{graphicx}
\newcommand{\ldt}{\bullet}
\newcommand{\into}{\hookrightarrow}
% Swap the definition of \abs* and \norm*, so that \abs
% and \norm resizes the size of the brackets, and the 
% starred version does not.
%\makeatletter
%\let\oldabs\abs
%\def\abs{\@ifstar{\oldabs}{\oldabs*}}
%
%\let\oldnorm\norm
%\def\norm{\@ifstar{\oldnorm}{\oldnorm*}}
%\makeatother
\newenvironment{subproof}[1][\proofname]{%
	\renewcommand{\qedsymbol}{$\blacksquare$}%
	\begin{proof}[#1]%
	}{%
	\end{proof}%
}

\usepackage{centernot}
\usepackage{dirtytalk}
\usepackage{calc}
\newcommand{\prob}[1]{\setcounter{section}{#1-1}\section{}}


\newcommand{\prt}[1]{\setcounter{subsection}{#1-1}\subsection{}}
\newcommand{\pprt}[1]{{\textit{{#1}.)}}\newline}
\renewcommand\thesubsection{\alph{subsection}}
\usepackage[sl,bf,compact]{titlesec}
\titlelabel{\thetitle.)\quad}
\DeclarePairedDelimiter\floor{\lfloor}{\rfloor}
\makeatletter	

\newcommand*\pFqskip{8mu}
\catcode`,\active
\newcommand*\pFq{\begingroup
	\catcode`\,\active
	\def ,{\mskip\pFqskip\relax}%
	\dopFq
}
\catcode`\,12
\def\dopFq#1#2#3#4#5{%
	{}_{#1}F_{#2}\biggl(\genfrac..{0pt}{}{#3}{#4}|#5\biggr
	)%
	\endgroup
}
\def\res{\mathop{Res}\limits}
% Symbols \wedge and \vee from mathabx
% \DeclareFontFamily{U}{matha}{\hyphenchar\font45}
% \DeclareFo\newcommand{\PP}{\mathbb{P}}ntShape{U}{matha}{m}{n}{
%       <5> <6> <7> <8> <9> <10> gen * matha
%       <10.95> matha10 <12> <14.4> <17.28> <20.74> <24.88> matha12
%       }{}
% \DeclareSymbolFont{matha}{U}{matha}{m}{n}
% \DeclareMathSymbol{\wedge}         {2}{matha}{"5E}
% \DeclareMathSymbol{\vee}           {2}{matha}{"5F}
% \makeatother

%\titlelabel{(\thesubsection)}
%\titlelabel{(\thesubsection)\quad}
\usepackage{listings}
\lstloadlanguages{[5.2]Mathematica}
\usepackage{babel}
\newcommand{\ffac}[2]{{(#1)}^{\underline{#2}}}
\usepackage{color}
\usepackage{amsthm}
\newtheorem{thm}{Theorem}[section]
\newtheorem*{thm*}{Theorem}
\newtheorem{conj}[thm]{Conjecture}
\newtheorem{cor}[thm]{Corollary}
\newtheorem{exle}[thm]{Example}
\newtheorem{lemma}[thm]{Lemma}
\newtheorem*{lemma*}{Lemma}
\newtheorem{problem}[thm]{Problem}
\newtheorem{prop}[thm]{Proposition}
\newtheorem*{prop*}{Proposition}
\newtheorem*{cor*}{Corollary}
\newtheorem{fact}[thm]{Fact}
\newtheorem*{prompt*}{Prompt}
\newtheorem*{claim*}{Claim}
\newcommand{\claim}[1]{\begin{claim*} #1\end{claim*}}
%organizing theorem environments by style--by the way, should we really have definitions (and notations I guess) in proposition style? it makes SO much of our text italicized, which is weird.
\theoremstyle{remark}
\newtheorem{remark}{Remark}[thm]
\newtheorem*{remark*}{Remark}

\theoremstyle{definition}
\newtheorem{defn}[thm]{Definition}
\newtheorem*{defn*}{Definition}
\newtheorem{notn}[thm]{Notation}
\newtheorem*{notn*}{Notation}
%FINAL
\newcounter{hwn}
\newcounter{ryr}
\setcounter{ryr}{226}
\newcommand{\course}{8254}
\newcommand{\due}{29 Vent\^ose \Roman{ryr}} 
\setcounter{hwn}{4}
\RequirePackage{geometry}
\geometry{margin=.7in}
\usepackage{todonotes}
\title{MATH \course~Homework \Roman{hwn}}
\author{David DeMark}
\date{\due}
\usepackage{fancyhdr}
\pagestyle{fancy}
\fancyhf{}
\rhead{David DeMark}
\chead{\due}
\lhead{MATH \course~ Homework \Roman{hwn}}
\cfoot{\thepage}
\renewcommand{\bar}{\overline}

% %%
%%
%%
%DRAFT

%\usepackage[left=1cm,right=4.5cm,top=2cm,bottom=1.5cm,marginparwidth=4cm]{geometry}
%\usepackage{todonotes}
% \title{MATH 8669 Homework 4-DRAFT}
% \usepackage{fancyhdr}
% \pagestyle{fancy}
% \fancyhf{}
% \rhead{David DeMark}
% \lhead{MATH 8669-Homework 4-DRAFT}
% \cfoot{\thepage}

%PROBLEM SPEFICIC
\renewcommand{\hom}{\mathrm{Hom}}
\newcommand{\lint}{\underline{\int}}
\newcommand{\uint}{\overline{\int}}
\newcommand{\hfi}{\hat{f}^{-1}}
\newcommand{\tfi}{\tilde{f}^{-1}}
\newcommand{\tsi}{\tilde{f}^{-1}}

\newcommand{\nin}{\centernot\in}
\newcommand{\seq}[1]{({#1}_n)_{n\geq 1}}
\newcommand{\Tt}{\mathcal{T}}
\newcommand{\card}{\mathrm{card}}
\newcommand{\setc}[2]{\{ #1\::\:#2 \}}
\newcommand{\idl}[1]{\langle #1 \rangle}
\newcommand{\cl}{\overline}
\newcommand{\id}{\mathrm{id} }
\newcommand{\im}{\mathrm{Im}}
\newcommand{\cat}[1]{{\mathrm{\bf{#1}}}}
%\usepackage[backend=biber,style=alphabetic]{biblatex}
%\addbibresource{algeo.bib}
\newcommand{\colim}{\varinjlim}
\newcommand{\clim}{\varprojlim}
\newcommand{\frp}{\mathop{\large {\mathlarger{\star}}}}
\newcommand{\restr}[2]{{\evat{#1}{#2}{}}}
\DeclareMathOperator{\codim}{codim}
\newcommand{\imp}[1]{\underline{#1}}
\newcommand{\ihm}{\imp{\hom}}
\newcommand{\him}{\ihm(\FF,\GG)}
\newcommand{\incla}{\hookrightarrow}
\newcommand{\pre}{\mathrm{pre}}
\newcommand{\Fp}{{\FF_P}}
\renewcommand{\thethm}{\arabic{section}.\Alph{thm}}
\newcommand{\gph}{\varphi}
\newcommand{\fv}[2]{\frac{x_{#1}}{x_{#2}}}
\newcommand{\va}{\vec{a}}
\newcommand{\vai}[1]{\va^{(#1)}}
\newcommand{\csch}{\cat{Sch}}
\newcommand{\cset}{\cat{Set}}
\newcommand{\aff}{\mathrm{aff}}
%\tikzcdset{column sep/tiny=.1cm}
\usepackage[backend=biber,style=alphabetic]{biblatex}
\addbibresource{algeo.bib}
\DeclareMathOperator{\proj}{Proj}
\newcommand{\tpsi}{{\tilde{\psi}}}
\newcommand{\surj}{\twoheadrightarrow}
\newcommand{\tvphi}{{\tilde{\varphi}}}
\newcommand{\hol}[2]{{#1}_{[#2]}}
\newcommand{\red}{\mathrm{red}}
\newcommand{\Rfr}{\mathfrak{R}}
\newcommand{\bph}{\overline{\phi}}
\begin{document}\maketitle
	\prob{1}
	\prob{2}
	\prob{3}
	\begin{prop*} For $R$, $S$ projective $A$-schemes,
		$\proj R\times_A\proj S\cong \proj (R\#_A S)$
	\end{prop*} \begin{proof}
\begin{lemma} For $f\in R$, $g\in S$, $e:=\abs{g}/\gcd(\abs{f},\abs{g})$, $d:=\abs{f}/\gcd(\abs{f},\abs{g})$,
	$((R\#_A)_{f^e\otimes g^d})_0\cong (R_f)_0\otimes_A (S_g)_0$.
\end{lemma}
\begin{subproof}
	We let $\phi:((R\#_A)_{f^d\otimes g^e})_0\cong (R_f)_0\otimes_A (S_g)_0$ map $(a\otimes b)/(f^e\otimes g^d)^k\mapsto \frac{a}{f^{ek}}\otimes \frac{b}{g^{dk}}$.
	
	 We first show that this map is well defined. We suppose $$\frac{a\otimes b}{(f^e\otimes g^d)^k}=\frac{c\otimes h}{(f^e\otimes g^d)^\ell}.$$
	Then, we have that there exists some $m$ such that $$(f^e\otimes g^d)^m\left((a\otimes b)(f^e\otimes g^d)^\ell-(c\otimes h)(f^e\otimes g^d)^k\right)=0$$
	i.e. $$af^{e(\ell+m)}\otimes bg^{d(m+\ell)}=cf^{e(k+m)}\otimes hg^{d(k+m)}$$
	in $R_{e(k+\ell+m)\abs{f}}\otimes S_{d(k+\ell +m)\abs{g}}$. Thus, there is some $r\in A$ such that $af^{e(\ell+m)}=rcf^{e(k+m)}$ and $rbg^{d(m+\ell)}=hg^{d(k+m)}$ modulo relabeling. Then, \begin{align*}
	\phi\left(\frac{a\otimes b}{(f^e\otimes g^d)^k}\right)=\frac{a}{f^{ek}}\otimes \frac{b}{g^{dk}}&=\frac{af^{e(\ell+m)}}{f^{e(k+\ell+m)}}\otimes \frac{bg^{d(\ell+m)}}{g^{d(k+\ell+m)}}\\
		&=\frac{rcf^{e(k+m)}}{f^{e(k+\ell+m)}}\otimes \frac{bg^{d(\ell+m)}}{g^{d(k+\ell+m)}}\\
	&=\frac{cf^{e(k+m)}}{f^{e(k+\ell+m)}}\otimes \frac{rbg^{d(\ell+m)}}{g^{d(k+\ell+m)}}\\
	&=\frac{cf^{e(k+m)}}{f^{e(k+\ell+m)}}\otimes \frac{hg^{d(k+m)}}{g^{d(k+\ell+m)}}\\
	&=\frac{c}{f^{e\ell}}\otimes \frac{h}{g^{d\ell}}=\phi\left(\frac{c\otimes h}{(f^e\otimes g^d)^\ell}\right),
	\end{align*}
	So indeed, our map is well-defined.

We now show injectivity. We suppose $$\phi\left(\frac{a\otimes b}{(f^e\otimes g^d)^k}\right)=\phi\left(\frac{c\otimes h}{(f^e\otimes g^d)^\ell}\right).$$
Then, we have that $$\frac{a}{f^{ek}}\otimes \frac{b}{g^{dk}}=\frac{c}{f^{e\ell}}\otimes \frac{h}{g^{d\ell}},$$ so as above, without loss of generality there is some $r\in A$ such that \begin{align*}
		&\frac{a}{f^{ek}}=r\frac{c}{f^{e\ell}}\\\text{and}\hspace{2em}&r\frac{b}{g^{dk}}=\frac{h}{g^{d\ell}}
	\end{align*}
	So we then have that there exist $m,n\in \NN$ such that \begin{align*}
	f^m(f^{e\ell}a-rcf^{ek})&=0\\
	g^n(g^{d\ell}rb-hg^{dk})&=0
	\end{align*}
	i.e.,
	\begin{align}
	\label{3eq1}f^{e(\ell+m}a&=rcf^{e(k+m)}\\
	\label{3eq2}g^{d(\ell+n)}rb&=hg^{d(k+n)}
	\end{align}
	We let $p=\max{(m,n)}$. Then, (\ref{3eq1},\ref{3eq2}) imply 
		\begin{align*}
	f^{e(\ell+p)}a&=rcf^{e(k+p)}\\
	g^{d(\ell+p)}rb&=hg^{d(k+p)}.
	\end{align*}
	Thus, we have that \begin{equation}\label{3segeq}\left(f^{e(\ell+p)}a\otimes g^{d(\ell+p)}b\right)-\left(f^{e(p+k)}c\otimes f^{e(p+k)}h \right)=0\end{equation} in $R\#_A S$. We may rephrase \eqref{3segeq} to \eqref{3pleq}
	\begin{equation}\label{3pleq}
		(f^e\otimes g^d)^p\left((a\otimes b)(f^e\otimes g^d)^\ell-(c\otimes h)(f^e\otimes g^d)^k\right)=0
	\end{equation}
%Tautologically, this implies \begin{equation}\label{3eq3}
%	f^{e(\ell+p)}a\otimes g^{d(\ell+p)}rb=rcf^{e(k+p)}\otimes hg^{d(k+p)}
%\end{equation}
%in $R\#_A S$. We reprhase \eqref{3eq3} to \eqref{3eq4}:
%\begin{equation}\label{3eq4}
%\left(f^e\otimes g^d\right)^p\left((f^{e}\otimes g^d)^\ell a\otimes rb\right)=\left(f^e\otimes g^d\right)^p\left(rcf^{ek}\otimes hg^{dk}\right)
%\end{equation}

 Finally, we pass through to the localization. By first principles of localization, \eqref{3pleq} implies \begin{equation*}
 \frac{a\otimes b}{(f^e\otimes g^d)^k}=\frac{c\otimes h}{(f^e\otimes g^d)^\ell}
 \end{equation*} as desired.
 
 Now, \textbf{finally,} we show surjectivity. We let $$r=\frac{a}{f^{dk}}\otimes \frac{b}{g^{e\ell}}\in (R_f)_0\otimes (S_g)_0$$ be arbitrary (making use of the equalities $R_f=R_{f^e}$, $S_g=S_{g^d}$ in order to write it in such a form). We let $m=\max(k,\ell)$ and then have that $$r=\frac{af^{m-k}}{f^{dm}}\otimes \frac{bf^{m-\ell}}{g^{em}}=\phi\left(\frac{af^{m-k}\otimes bf^{m-\ell}}{(f^e\otimes g^d)^m}\right).$$ This completes our grotesquely ugly proof.
\end{subproof}
To see that the main proposition follows from the lemma, we note that the isomorphism of rings given in the lemma gives an isomorphism of open affine subschemes $D_+(f^e\otimes g^d)\cong D_+(f)\times_AD_+(g)$. As $\{D_+(f^e\otimes g^d)\}_{f,g\in R\times S}$ is an open affine cover of $\proj(R\#_A S)$ and $\{D_+(f)\times_AD_+(g)\}_{f,g\in R\times S}$ gives an open affine cover of $\proj R\times_A\proj S$, the isomorphisms on open affines glue to an isomorphism of schemes as desired.
\end{proof}
	\prob{4}
	\prob{5}We let $S:=k[x,y]$ and $R:=k[u,v,w]/\idl{uv-w^2}$
	\begin{prop*}
		$S_{2\ldt}\cong R$ by the map $\phi:R\to S_{2\ldt}$ mapping $u\mapsto x^2$, $v\mapsto y^2$, $w\mapsto xy$. 
	\end{prop*}
\begin{proof}
We define a map $\bph: k[u,v,w]\to S_{2\ldt}$ by $u\mapsto x^2$, $v\mapsto y^2$, $w\mapsto xy$. Then, $(uv-w^2)\mapsto x^2y^2-(xy)^2=0$, so $\bph$ induces a well-defined map $\phi:R\to S_{2\ldt}$ as described in the proposition. To show surjectivity, we consider a monomial $m(x,y)=x^{2m-k}y^k\in S_{2m}$. If $k$ is even, we write $k=2j$ and have that $m(x,y)=(x^2)^{m-j}y^j=\phi(u^{m-j}v^{j})$. If $k$ is odd, we write $k=2j-1$ and have that $m(x,y)=(x^2)^{m-j}(y^2)^{j-1}(xy)=\phi(u^{m-j}v^{j-1}w)$. Thus, as $\phi$ surjects from a subset of $R$ to a generating set of $S_{2\ldt}$ as a $k$-module, we have that $\phi$ is indeed surjective. Finally, to show injectivity, we construct an explicit splitting map $\psi$, then show that $\psi$ is indeed surjective. We map \begin{equation*}\psi: \left(x^{2m-k}y^k\in S_{2m}\right)\mapsto \begin{cases}
u^{2m-2j}v^j&k=2j\\
u^{m-j}v^{j-1}w&k=2j-1.
\end{cases}\end{equation*} Inspection immediately shows that $\phi\circ \psi=\id_{S_{2\ldt}}$. We let $m(u,v,w)=u^av^bw^c\in R_m$ and write $c=2d+\delta$ with $\delta\in \{0,1\}$. Then, $u^av^bw^c-u^{a+d}v^{b+d}w^\delta=u^av^bw^\delta(w^{2d}-u^dv^d)\in \idl{uv-w^2}$, so we have that an arbitrary monomial $m(u,v,w)$ is indeed in the image of $\psi$ modulo $\idl{uv-w^2}$. We have now completed our proof.
\end{proof}
	\prob{6}
	\prob{7}
	\prob{8}
\end{document}